\section{Testing and results}
No formal testing was preformed as the code was clear and it was easy to maintain control over what the code did.
Ad-hoc testing was conducted when applicable to supplement this.
To familiarize our selves with the supplied code, the LED were tested with a simple example.

To mesure difference in energy efficiency a loop was implemented at first. 
This loop was then used to control the GPIO inputs. Later, the same task was implemented using interupts.
The results are rather interesting in terms of power consumption compared to the test code from exercise-0.
The power consuption had an avarage of 3.59 mA while using only looping without interupts. 
When using interupt for the GPIO handler without using any sleep modes, the power consuption fell by around 4 \% to 3.44 mA.
To reduce the power consuption further normal sleep was introduced. The result of this was as little as 1.25 mA, or down 64 \%.
The last test was preformed with deep sleep. The power consumption was then at 1.7 $\my$A. 
This illustrates how much relative power that can be saved. 
As this was a simple demo there wasn't any obvious disadvantages by using deepsleep while the benefits was enourmous.
