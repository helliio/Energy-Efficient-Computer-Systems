\section{Testing and results}
No formal testing was preformed, the team conducted more of a trial-and-error kind of approach.
The LEDs were tested first to familiarise ourselves with the supplied code and the microcontroller.

After some testing with setting the LED register and using lsl and lsr to left and right shift, we were ready to implement the buttons.

The most intuitive method was used at first.
Instead of using interrupts a loop was used instead.
The button 7 and button 5 would left and right shift the row of lights, in other words the button 5 would fill up the bar from the right and the button 7 would empty it.
The results are rather interesting in terms of power consumption compared to the test code from exercise-00.
The power consumption was averaging at 3.42 mA while idling and looping. 
The power consumption while listening for buttons was at 3.59 mA.
A button press increased the power consumption by 0.04 mA.

Next it is the time to implement the code with interrupts.
The code was modified so that instead of looping we would use the gpio handler.
The first test was idle power consumption without any sleep, the results were at 3.44 mA.
The second test was with normal sleep.
The power consumption was at 1.25 mA.
The last test was preformed with deep sleep.
The power consumption was at 1.7 uA with none of the leds on and 3.8 uA with all the leds on.

The results looked much more promising.
This was much much more power efficient than the looping.
