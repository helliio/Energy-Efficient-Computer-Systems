\documentclass[a4,journal,twoside]{IEEEtran}
\usepackage{blindtext}
\usepackage{graphicx}
\usepackage{subfiles}
\usepackage[detect-all]{siunitx}
\usepackage{minted}
\usepackage{todonotes}
\usepackage{hyperref}

\usepackage[listings,skins]{tcolorbox}

\usepackage{amsmath}
\usepackage{mathtools}
\usepackage{amsfonts}

\usepackage[utf8]{inputenc}

\title{
    \textsc{Linux and Drivers on ARM Cortex-M3}\\
    Exercise 3 in TDT4258
}

\author{
    Aleksander~\textsc{Wasaznik},
    Geir~\textsc{Kulia},
    Liang~\textsc{Zhu}
}

\markboth{Exercise 3, TDT4258 Energy Efficient Computer Systems}{}

\begin{document}\maketitle

\begin{abstract}
The objective of this exercise is to give a better understanding of linux device drivers. This was done by developing a black and white version of the classic game Pong\cite{pong}. The game was controlled by the gamepad used in previous exercises\cite{ex1}\cite{ex2}.

To be able to interact with the gamepad a device driver was implemented. This device driver is dependent on the hardware of the ARM Cortex-M3 and is Linux-spesific. It works as a program that operates the interaction between the and the software on the ARM Cortex-M3 and the gamepad.
\end{abstract}

% Chapters go here.
\subfile{sections/introduction.tex}
\subfile{sections/development.tex}
\subfile{sections/conclusion.tex}
\subfile{bib.tex}

\end{document}
