\documentclass[../main.tex]{subfiles}
\section{Introduction}
The main goal of this excercise was to implement a driver for Linux for the same gamepad used in the previous excercises\cite{ex1}\cite{ex2}.
The driver should have the features nessesary for the game, also developed for this excercise.

\subsection{Pong}
The choice of game was the classic, Pong.
Pong is one of the first arcade games ever made.
The game is a minimalistic game of table tennis in two dimensions, with a top view.
Two players each have a pad on oposing edges of the screen that can be moved along the edge.
They compete to knock a ball past the pad of the oponent using their pad.
The ball bounces against the non-player sides and has constant velocity when not bouncing.
The game is thus more akin to air hockey with a large goal than table tennis.
See Figure \ref{fig:pong} for an illustration.

\subsection{What the game is structured around}
When developing a game there is a need to identify what external factors drive it.
While in play, the ball moves as time passes
and the pads move as time passes depending on the input for the players.
In this state, the game clearly only advances on time intervals.
The game thus needs to have access to the state of the gamepad buttons every time the game state progresses.
This is a good fit for simply polling the driver when input is needed.
If the cost of the syscalls dominates, using input-change interrupts might help.

The game will need to display its graphics on the screen on the development board.

\subsection{Gamepad driver}
In the previous excercises, the programs accessed hardware periperals directly.
However, unlike the situation in the previous excercises, the program now runs under the Linux kernel.
It is the job of the kernel to manage shared resources, such as processor, memory and other hardware.
User space programs such as the game should be unable to interfere with one another
, and should access hardware though a safe and abstracted interface.

It is the role of a device driver to mediate access to hardware for unprivileged programs
, and ensure that multiple programs concurrently wishing for access do not interfere with each other.
Drivers are executed in kernel mode, allowing access to hardware.
The driver communicates with user-space programs by exposing \emph{device nodes} which special files in the filesystem can be associated with.

At its most basic, the driver for the gamepad should support reading the current state of the buttons in some way.
Favouring simplicity, the driver should just yield a byte that is a bitmask over the button positions.
