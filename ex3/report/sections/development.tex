\documentclass[../main.tex]{subfiles}
\section{Development}

\subsection{Linux}
ptxdist build system
testing the handed out files

the components of a linux system
bootloader
kernel
root filesystem

\subsection{Framebuffer}
Linux takes inspiration from UNIX in that a lot of resources appear as files in filesystem.
Indeed, the contents of a screen can be manipulated through a file with a framebuffer device.
The framebuffer file is a character file which contains the pixel values on the screen in left-to-right downwards order.
The way color is represented varies from screen to screen.
In this case the pixels are represented by 5-6-5 bits of red-green-blue.

There is only one screen on the development board.
It is the first framebuffer device with mayor and minor numbers 29 0.
A file for the framebuffer can be found at \texttt{/dev/fb0}.

\subsubsection{Using the Framebuffer}
As the framebuffer is a file, changing the contents of the screen is as simple as writing to the file.
For testing, a program that shows a color pattern on the screen by writing to the file using normal file operations.
The program \texttt{open}ed the file for writing, and filled the screen with one \texttt{write} per pixel.
The result was dismaying, as only a few lines at the top of the screen were altered repeatedly.
It would seen as if the position in the file gets reset after too many writes.
Since using the framebuffer file with file opeartions is apparently unusual, there were no hints as how to solve this.
Filling the whole screen in one write operation did work, however that requires a buffer as large as framebuffer in memory.
That is neither optimal, nor will it fit on the stack.
Trying to write to a buffer this size on the stack, just crashes the operating system.

The most common and convenient way to use the framebuffer is to map its contents to memory.
This is accomplished by opening the framebuffers character file.
\texttt{mmap} needs it to be opened in read-write mode even if only writing is desired.
Then a shared memory mapping needs to be establised.
The call to \texttt{mmap} will return the address at which the contents of the file can be found.

\begin{minted}{c}
fbfd = open("/dev/fb0", O_RDRW);
fd = mmap(NULL, fb_length, PROT_WRITE,
          MAP_SHARED, fbfd, 0);
\end{minted}

Letting fb have the type

\begin{minted}{c}
pixel_t (*fb)[screen_height][screen_width]
\end{minted}

allows the screen to be updated conveniently by

\begin{minted}{c}
(*fb)[y][x] = color
\end{minted}

The particular driver for the screen on the EFM32GG needs to be told to update the screen explicitly.
Presumably because updating in bulks is more efficient.
The way to request a screen update is to issue \texttt{ioctl} 0x4680 along with a datastructure describing the screen location of the update.

The mmap aproach was chosen for the game because it worked right away and was most conveient.

\subsection{Static Scene}
First, a program rendering a still scene of the game was made.
The graphics of Pong consists solely of rectangular shapes.
The two rectangular pads and the square ball.
A function for drawing such a primitive was written.
The function takes a struct describing the width, height and coordinates of a rectangle.
The scene is thus described in three statically allocated structs.

\subsection{Animation}
Next, the ball was made to move across the screen.
A naive aproach to rendering graphics is to redraw the whole screen at each frame.
This proved to be rather slow.
There are two separate but related processes that need to be sped up;
The rendering and transfering to screen.

It is cheaper to render just the difference between two frames of animation.
The structs describing the rectangles were altered to hold the coordinates the object was previously rendered at.
This allows a rendering function to render only the difference.
The assumption that two objects will never overlap allows each object to still be treated independently, not needing redrawing on each frame.
However, to avoid artefacts when an objects rendered first moves into the previous position of an object erased later, all erazures must happens before drawing.

It became apparent that transfering to screen was the most costly operation.
To do less updates, bounding boxes for changes were provided to the driver, giving a great speedup.
To avoid flickering, it is important that only parts of the screen that have finished rendering all objects are updated.

When animating, it is important that the speed of animation is independent of the workload or complexity of the scene.

\subsection{Physics}
The physics of the game are simple and consist of moving the ball and making it bounce of walls and pads upon contact, calculating the angle of bounce on a pad depending on where the ball hit the pad.
The accuracy of such physics depends on how small the animations steps are.
It was found that the game ran fast enough for single pixel movement.

\subsection{Driver}
The driver reserves and initializes the hardware upon load.
It then registers a character device.
Whenever a read is performed on the character device
, the driver reads the buttons pressed from the GPIO hardware registers
and returns it as a bitmap integer.

The reserving is done by means of reserve\_region.
This function does not have any purpose, but to deny two reservations of the same address space.
A good question is then what address to reserve as a token to prevent interference.
Ideally, we would like to have exclusive access to control PC pins 0-7.
It is also acceptable to take the whole PC, which is what we choose to do.
If interrupts are to be used, the interrupt control register should also be taken control over.

The character device is created by registering a device number:
\begin{minted}{c}
dev_t dev = MKDEV(MAJOR_NUMBER, 0);
register_chrdev_region(dev, 1, NAME);
\end{minted}

and telling Linux about it:

\begin{minted}{c}
cdev_init(&cdev, &fops);
cdev.owner = THIS_MODULE;
cdev_add(&cdev, dev, 1);
\end{minted}
