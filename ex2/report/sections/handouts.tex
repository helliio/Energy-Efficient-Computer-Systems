\section{Handed-out Materials}
The handed-out framework comes with a convenient makefile.
The makefile had targets for building the file for uploading, the ELF file used with GDB and for actually uploading the code to the microcontroller.
Since we are going to do the coding in the programming languge C we were given a \emph{efm32gg.h} file that contained all the registres that we will need in this exsercise.
The skeleton code is spread across 4 \emph{.c} files.
The files were \emph{dac.c} \emph{ex2.c} \emph{gpio.c} \emph{interrupt_handler.c} and \empth{timer.c}.
The \empth{dac.c} file has the sole purpose of enabling the digital/analouge converter.
\empth{timer.c} is a file that enables the timer interupts.
\empth{gpio.c} is a file that initializes the GPIO registers.
\empth{interrupt_handler.c} is where we handle the GPIO interrupts and the timer interuptes.
Finally the \empth{ex2.c} is where the main function is and where the initialisers where ran.
