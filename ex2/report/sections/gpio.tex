\documentclass[../main.tex]{subfiles}
\section{GPIO}
The function \emph{setupGPIO()} will be called in the main function to activete GPIO funtionalities.
The implementation of the function is located in the /emph{setup.c} file.

\begin{minted}{objdump}
void setupGPIO()
{
    *CMU_HFPERCLKEN0   |=   CMU2_HFPERCLKEN0_GPIO; /* enable GPIO clock*/
    *GPIO_PA_CTRL       =   2;  /* set high drive strength */
    *GPIO_PA_MODEH      =   0x55555555; /* set pins A8-15 as output */
    *GPIO_PA_DOUT       =   0x0000; /* turn on LEDs D1-D8 (LEDs are active low) */
    *GPIO_PC_MODEL      =   0x33333333; /* Set pins 0-7 */
    *GPIO_PC_DOUT       =   0xff; /* Enable internal pull-up */
    *GPIO_EXTIPSELL     =   0x22222222; /* Write 0x22222222 to register */
    *GPIO_EXTIFALL      =   0xff; /* Set interrupt on 1->0 transition */
    *GPIO_IEN           =   0xff; /* Enable interrupt generation */
}
\end{minted}

After the GPIO setup is completed, when ever a button is pressed, the /emph{handle\_gpio()} function i s called and the desired code will be executed.
