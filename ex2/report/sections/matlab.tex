\documentclass[../main.tex]{subfiles}
\section{Piano Sample}
A sample was used to simulate a piano tone. Another approach that was discussed was to generate the sound from the Fourier transform. To use a sample was preferable because it is less power consuming to look up a table than to calculate a waveform based on a Fourier transform as this is a more costly operations for the ALU, taking several cycles. 

A recording of a piano playing the tone C was used to generate the signal. A MatLab then converted the script to an array. To save space at the microcontroller the sampling frequency was reduced by a factor on eight and the reverberation was drastically reduced. Values repeating themselves in the beginning and of the signal were also removed. 

The convert the signal to match the values used by DAC in the c program on the microcontroller it was necessary to increase the numbers $\in [-1, 1]$ to $\in [0x000, 0xfff]$. This was done with equation %\ref{EQUATION}.

\begin{equation}
y_{dac} = \frac{(x_{mp3} + 1)}{2} \cdot 2^{12}
\end{equation}

Here $x_{mp3}$ is the signal loaded from the mp3-file and $y_{dac} \in \mathbb{Z}_{\text{0xfff}}$  is a signal with corresponding values to match the DAC. The MatLab code below shows how this was implemented.
\\ \\

\begin{minted}{matlab}

y = y(y ~= 0); % Remove all zeros

% Decrese Fs and reverberation
y = y(1:8:length(y)/5); 

y = round((y + 1) .* 2^11);
y = y(y ~= 2048); % remove 0x800 bec. Repeat

\end{minted}


Then the array was formatted and exported to a header file. 

